%*******************************************************************************
%*********************************** First Chapter *****************************
%*******************************************************************************
\chapter{Introduction}  %Title of the First Chapter
\label{chapter1}
\ifpdf
    \graphicspath{{Chapter1/Figs/Raster/}{Chapter1/Figs/PDF/}{Chapter1/Figs/}}
\else
    \graphicspath{{Chapter1/Figs/Vector/}{Chapter1/Figs/}}
\fi


%********************************** %First Section  **************************************

\section{Inquadramento generale}
La prima parte contiene una frase che spiega l'area generale dove si svolge il lavoro; una che spiega la sottoarea pi\`u specifica dove si svolge il lavoro e la terza, che dovrebbe cominciare con le seguenti parole ``lo scopo della tesi \`e \dots'', illustra l'obbiettivo del lavoro. Poi vi devono essere una o due frasi che contengano una breve spiegazione di cosa e come \`e stato fatto, delle attivit\`a  sperimentali, dei risultati ottenuti con una valutazione e degli sviluppi futuri. La prima parte deve essere circa una facciata e mezza o due.
The ultimate goal is to help
children with autism in making sense of the world, assisted by computer and robotic technology.

\section{Breve descrizione del lavoro}
La seconda parte deve essere una esplosione della prima e deve quindi mostrare in maniera pi\`u esplicita l'area dove si svolge il lavoro, le fonti bibliografiche pi\`u importanti su cui si fonda il lavoro in maniera sintetica (una pagina) evidenziando i lavori in letteratura che presentano attinenza con il lavoro affrontato in modo da mostrare da dove e perch\'e \`e sorta la tematica di studio. Poi si mostrano esplicitamente le realizzazioni, le direttive future di ricerca, quali sono i problemi aperti e quali quelli affrontati e si ripete lo scopo della tesi. Questa parte deve essere piena (ma non grondante come la sezione due) di citazioni bibliografiche e deve essere lunga circa 4 facciate.

\section{Struttura della tesi}
La terza parte contiene la descrizione della struttura della tesi ed \`e organizzata nel modo seguente.
``La tesi \`e strutturata nel modo seguente.

Nella sezione due si mostra \dots

Nella sez. tre si illustra \dots

Nella sez. quattro si descrive \dots

Nelle conclusioni si riassumono gli scopi, le valutazioni di questi e le prospettive future \dots

Nell'appendice A si riporta \dots (Dopo ogni sezione o appendice ci vuole un punto).''

I titoli delle sezioni da 2 a M-1 sono indicativi, ma bisogna cercare di mantenere un significato equipollente nel caso si vogliano cambiare. Queste sezioni possono contenere eventuali sottosezioni.


% *************************************
\section{Thesis Outline} %Section - 1.2
% *************************************
\paragraph{Chapter \ref{chapter2}} outlines the main concepts used in the thesis, which range from basic introduction to holonomic robots to the description of various autonomous navigation and obstacle avoidance techniques used for such robots. 
\paragraph{Chapter \ref{chapter3}} introduces the adopted holonomic platform including measures, hardware and software details concerning the internal ROS infrastructure.
\paragraph{Chapter \ref{chapter4}} describes relevant related works on the activity recognition and Robogame field. The environment in which the omni-directional platform is used is presented and a general description of the Robogame is given, with the adopted game's rules. We finally investigate the possibility of human physical activity recognition in a robot game scenario and explain how is possible to use activity recognition techniques to enable robot behavior adaptation to support player engagement during the game.
\paragraph{Chapter \ref{chapter5}} introduces the adopted control scheme for in-game holonomic navigation and obstacle avoidance and describe the Matlab-ROS interface used during the testing phase.


